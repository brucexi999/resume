\documentclass[11pt,a4paper,sans]{moderncv}

% moderncv themes
\moderncvstyle{banking}
\moderncvcolor{blue}

% adjust the page margins
\usepackage{lmodern}
\usepackage[scale=0.8, top=1cm, bottom=1cm, left=1cm, right=1cm]{geometry}

% personal data
\name{Bruce}{Xi}
\phone[mobile]{(236)-777-8218}
\email{brucexi99@outlook.com}
\social[linkedin]{bruce-shidi-xi}
\social[github]{brucexi999}
%\extrainfo{\textbf{Available from May 2024 - August 2025}}

%----------------------------------------------------------------------------------
%            content
%----------------------------------------------------------------------------------
\begin{document}
\makecvtitle

\section{Education}
\cventry{2021--Present}{Master of Engineering in Electrical and Computer Engineering}{The University of British Columbia}{Vancouver, BC}{GPA 4.0}{
    \begin{itemize}
    \item Relevant courses: Deep Learning, Reinforcement Learning, ML Hardware Accelerator, Computer Architecture, Digital/Microcomputer System Design, VLSI, IC Testing and Reliability
\end{itemize}
}
\cventry{2018--2021}{Bachelor of Engineering in Materials Science and Engineering}{Imperial College London}{London, UK}{}{
    \begin{itemize}
        \item Graduated with First-Class Honours
        \item Obtained Dean's List for three consecutive years (2018-2021)
    \end{itemize}
}

\section{Experience}
\cventry{May-Dec. 2022}{Design Validation Co-op}{Motorola Solutions}{Vancouver, BC}{}
{
    \begin{itemize}
    \item Conducted extensive surveillance camera tests in various settings, validating the cameras' electrical, mechanical, and optical performance
    \item Engineered Python-based software for test automation and data analysis, enhancing test efficiency with up to 90\% automation
    \item Utilized FFmpeg for video processing and analysis
    \item Collaborated within a team using Git, Confluence, and Jira to optimize workflow efficiency
    \end{itemize}
}

\section{Project}

\cventry
{\textnormal{\textbf{May-Oct. 2023}}}
{\textnormal{\textbf{Concurrent VLSI Routing with Multi-agent
Deep Reinforcement Learning}}}
{}{}{}
{
    \begin{itemize}
        \item Developed a Python-based machine learning framework to address the VLSI global routing problem in a concurrent manner. This framework modeled routing as a pathfinding task and solved it using multi-agent reinforcement learning integrated with deep neural networks (MLP and GNN implemented in PyTorch)
        \item Addressed training challenges by fine-tuning hyperparameters through a grid search approach, leading to significant performance improvements
        \item The proposed work overcame the traditional net-ordering issue, guaranteed zero overflow, and outperformed an A* baseline by 2.6\% in terms of wirelength
        \end{itemize}
}

\cventry
{\textnormal{\textbf{Sept.-Dec. 2023}}}
{\textnormal{\textbf{Computer Architecture Simulator}}}
{}{}{}
{
    \begin{itemize}
        \item Leveraged tiling and loop reordering to optimize matrix multiplication in C.These techniques changed memory access patterns and improved cache hit rate by 8x
        \item Developed four cache replacement policies in C++ for the ChampSim simulator, achieving Instruction Per Cycle (IPC) and hit rates comparable to the Least Recently Used (LRU) policy
        \item Implemented two branch prediction algorithms in C++, resulting in IPC and accuracies on par with a established baseline
    \end{itemize}
}

\cventry
{\textnormal{\textbf{Jan.-Apr. 2023}}}
{\textnormal{\textbf{Microcomputer System Development}}}
{}{}{}
{
    \begin{itemize}
    \item Engineered key components of an microcomputer system including a 4-way set-associative cache and a DRAM controller using Verilog. The cache reduced the runtime of a benchmark by 43\%
    \item Implemented the system on an Altera FPGA with a soft core provided by the course 
    \item Developed software and firmware in C that interacted with hardware (Flash, EEPROM, and ADC/DAC) using SPI, IIC, and CAN protocol 
    \item Utilized hardware timer interrupt and designed a snake game software that ran on the system and interact with the player using VGA
    \end{itemize}
}

\cventry
{\textnormal{\textbf{June-Sept. 2022}}}
{\textnormal{\textbf{CPU Design and Assembly programming}}}
{}{}{}
{
    \begin{itemize}
    \item Architected a 16-bit RISC CPU from the ground up using Verilog, integrating pivotal components such as FSM, datapath, RAM, and I/O interfaces. The CPU was implemented on an Altera FPGA
    \item The CPU supported 13 diverse instructions encompassing ALU operations, memory access, and branching mechanisms
    \item Implemented preemptive multitasking on a ARM core integrated on the FPGA using Assembly
\end{itemize}
}

\section{Skills}
\cvitem{Hardware}{Verilog, FPGA, ModelSim, Quartus, Cadence}
\cvitem{Software}{Python $>$ C $>$ ARM Assembly = C++, Linux}
\cvitem{DevOps Tools}{Confluence, Jira, Git, GitHub}

% ... add more sections and subsections as needed ...
\end{document}
