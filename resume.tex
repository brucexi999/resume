\documentclass[11pt,a4paper,sans]{moderncv}

% moderncv themes
\moderncvstyle{banking}
\moderncvcolor{blue}

% adjust the page margins
\usepackage{lmodern}
\usepackage[scale=0.8, top=1.5cm, bottom=1.5cm, left=2cm, right=2cm]{geometry}

% personal data
\name{Bruce (Shidi)}{Xi}
\address{303-4240 Cambie St, Vancouver, BC, V5Z 2Y4}
\phone[mobile]{(236)-777-8218}
\email{brucexi99@outlook.com}
\social[linkedin]{bruce-shidi-xi}
\social[github]{brucexi999}

%----------------------------------------------------------------------------------
%            content
%----------------------------------------------------------------------------------
\begin{document}
\makecvtitle

\section{Education}
\cventry{2021--2024}{Master of Engineering in Electrical and Computer Engineering}{The University of British Columbia}{Vancouver, BC}{}{
    \begin{itemize}
    \item Relevant courses: Deep Learning, ML Hardware Accelerator, Computer Architecture, Digital System Design, Embedded System, VLSI, IC Testing and Reliability
\end{itemize}
}  % arguments 3 to 6 can be left empty
\cventry{2018--2021}{Bachelor of Engineering in Materials Science and Engineering}{Imperial College London}{London, UK}{}{
    \begin{itemize}
        \item Graduated with First-Class Honours
        \item Obtained Dean's List for three consecutive years (2018-2021)
    \end{itemize}
}

\section{Project}

\cventry
{\textnormal{\textbf{May-Oct. 2023}}}
{\textnormal{\textbf{Multi-agent Deep Reinforcement Learning for VLSI Routing}}}
{}{}{}
{
    \begin{itemize}
    \item Demonstrated initiative by independently mastering DRL and VLSI global routing through exhaustive self-study and comprehensive literature review
    \item Successfully modeled and implemented the physical routing problem in Python
    \item Designed an innovative multi-agent DRL model to address the routing challenge, leveraging the sophisticated capabilities of Python libraries
\end{itemize}
}

\cventry
{\textnormal{\textbf{Jan.-Apr. 2023}}}
{\textnormal{\textbf{Embedded System Design}}}
{}{}{}
{
    \begin{itemize}
    \item Designed key components of an embedded system including a 4-way set-associative cache controller and a DRAM controller using Verilog. The cache reduced the runtime of a benchmark by 43\%
    \item Implemented the system on an FPGA with a provided soft microcontroller 
    \item Developed software and firmware in C that interacted with hardware using SPI, IIC, and CAN protocol 
    \item Utilized hardware timer interrupt and designed a snake game software that ran on the embedded system
    %\item Project repository and demonstration videos can be found on GitHub (repo name: UBC-CPEN-412)
\end{itemize}
}

\cventry
{\textnormal{\textbf{June-Sept. 2022}}}
{\textnormal{\textbf{CPU Architecture Design}}}
{}{}{}
{
    \begin{itemize}
    \item Architected and crafted a 16-bit RISC CPU from the ground up using Verilog, integrating pivotal components such as FSM, datapath, RAM, and I/O interfaces
    \item The CPU supported 13 diverse instructions encompassing ALU operations, memory access, and branching mechanisms
    \item Successfully deployed the system onto an FPGA and validated the design's capabilities by executing a test program
\end{itemize}
}

\section{Experience}
\cventry{May-Dec. 2022}{Design Validation Co-op}{Motorola Solutions}{Vancouver, BC}{}
{
    \begin{itemize}
    \item Orchestrated a comprehensive range of camera tests, ensuring precision both in lab settings and office environments
    \item Developed Python-based software, realizing test automation and data analysis, resulting in a significant enhancement in test efficiency. Some tests achieved automation of up to 90\%
    \item Collaborated effectively within a team framework, leveraging tools like Git, Confluence,  and Jira for optimal workflow management 
    \end{itemize}
}

\section{Skills}
\cvitem{Hardware}{Verilog, Assembly, FPGA, ModelSim, Quartus, Cadence}
\cvitem{Software}{Python, C, Linux, Deep Learning, Reinforcement Learning}
\cvitem{yo?}{Confluence, Jira, Git, GitHub}
\cvitem{}{}

% ... add more sections and subsections as needed ...
\end{document}
