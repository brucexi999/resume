\documentclass[11pt,a4paper,sans]{moderncv}

% moderncv themes
\moderncvstyle{banking}
\moderncvcolor{blue}

% adjust the page margins
\usepackage{lmodern}
\usepackage[scale=0.8, top=1.5cm, bottom=1.5cm, left=2cm, right=2cm]{geometry}

% personal data
\name{Bruce}{Xi}
\phone[mobile]{(236)-777-8218}
\email{brucexi99@outlook.com}
\social[linkedin]{bruce-shidi-xi}
\social[github]{brucexi999}

%----------------------------------------------------------------------------------
%            content
%----------------------------------------------------------------------------------
\begin{document}
\makecvtitle

\section{Education}
\cventry{2021--2024}{Master of Engineering in Electrical and Computer Engineering}{The University of British Columbia}{Vancouver, BC}{GPA 4.0}{
    \begin{itemize}
    \item Relevant courses: Deep Learning, ML Hardware Accelerator, Computer Architecture, Digital System Design, Embedded System, VLSI, IC Testing and Reliability
\end{itemize}
}
\cventry{2018--2021}{Bachelor of Engineering in Materials Science and Engineering}{Imperial College London}{London, UK}{}{
    \begin{itemize}
        \item Graduated with First-Class Honours
        \item Obtained Dean's List for three consecutive years (2018-2021)
    \end{itemize}
}

\section{Experience}
\cventry{May-Dec. 2022}{Design Validation Co-op}{Motorola Solutions}{Vancouver, BC}{}
{
    \begin{itemize}
    \item Conducted extensive camera tests in various settings, ensuring accuracy and precision
    \item Engineered Python-based software for test automation and data analysis, enhancing test efficiency with up to 90\% automation
    \item Collaborated within a team using Git, Confluence, and Jira to optimize workflow efficiency
    \end{itemize}
}

\section{Project}

\cventry
{\textnormal{\textbf{May-Oct. 2023}}}
{\textnormal{\textbf{Multi-agent Deep Reinforcement Learning for VLSI Routing}}}
{}{}{}
{
    \begin{itemize}
    \item Self-taught DRL and VLSI global routing, demonstrating initiative and commitment
    \item Developed a multi-agent DRL router using PyTorch and RLlib, the router was equipped with a custom GNN for enhanced policy generalization
    \item The proposed router outperformed an A* baseline by 2.6\%
\end{itemize}
}

\cventry
{\textnormal{\textbf{Jan.-Apr. 2023}}}
{\textnormal{\textbf{Embedded System Development}}}
{}{}{}
{
    \begin{itemize}
    \item Engineered key components of an embedded system including a 4-way set-associative cache controller and a DRAM controller using Verilog. The cache reduced the runtime of a benchmark by 43\%
    \item Implemented the system on an FPGA with a provided soft microcontroller 
    \item Developed software and firmware in C that interacted with hardware using SPI, IIC, and CAN protocol 
    \item Utilized hardware timer interrupt and designed a snake game software that ran on the embedded system
    \end{itemize}
}

\cventry
{\textnormal{\textbf{June-Sept. 2022}}}
{\textnormal{\textbf{CPU Architecture Design}}}
{}{}{}
{
    \begin{itemize}
    \item Architected a 16-bit RISC CPU from the ground up using Verilog, integrating pivotal components such as FSM, datapath, RAM, and I/O interfaces
    \item The CPU supported 13 diverse instructions encompassing ALU operations, memory access, and branching mechanisms
    \item Successfully deployed the system onto an FPGA and validated the design's capabilities by executing a test program
\end{itemize}
}

\cventry
{\textnormal{\textbf{Jan.-Apr. 2022}}}
{\textnormal{\textbf{System-on-Chip Design}}}
{}{}{}
{
    \begin{itemize}
    \item Designed an SoC which was implemented on an FPGA using Quartus IP Catalog, the system consisted of a soft CPU, an on-chip memory, a memory-mapped interconnect, and I/Os
    \item Designed custom hardware using Verilog that adhered to the interconnect's communication protocol 
    \item Scripted C programs to interact with the system
    \end{itemize}
}

\section{Skills}
\cvitem{Hardware}{Verilog, FPGA, ModelSim, Quartus, Cadence}
\cvitem{Software}{Python, C, Assembly, Linux}
\cvitem{DevOps Tools}{Confluence, Jira, Git, GitHub}

% ... add more sections and subsections as needed ...
\end{document}
