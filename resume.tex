\documentclass[11pt,a4paper,sans]{moderncv}

% moderncv themes
\moderncvstyle{banking}
\moderncvcolor{blue}

% adjust the page margins
\usepackage{lmodern}
\usepackage[scale=0.8, top=1cm, bottom=1cm, left=1cm, right=1cm]{geometry}

% personal data
\name{Bruce}{Xi}
\phone[mobile]{(236)-777-8218}
\email{brucexi99@outlook.com}
\social[linkedin]{bruce-shidi-xi}
\social[github]{brucexi999}
%\extrainfo{\textbf{Available from May 2024 - August 2025}}

%----------------------------------------------------------------------------------
%            content
%----------------------------------------------------------------------------------
\begin{document}
\makecvtitle

\section{Education}
\cventry{2021--Present}{Master of Engineering in Electrical and Computer Engineering}{The University of British Columbia}{Vancouver, BC}{GPA 4.0}{
    \begin{itemize}
    \item Relevant courses: Digital/Microcomputer System Design, Computer Architecture, VLSI, IC Testing and Reliability, DNN Accelerator, Deep Learning
\end{itemize}
}
\cventry{2018--2021}{Bachelor of Engineering in Materials Science and Engineering}{Imperial College London}{London, UK}{First-Class Honours Degree}{
    \begin{itemize}
    %    \item Graduated with First-Class Honours
        \item Obtained Dean's List for three consecutive years (2018-2021)
    \end{itemize}
}

\section{Experience}
\cventry{May-Dec. 2022}{Design Validation Co-op}{Motorola Solutions}{Vancouver, BC}{}
{
    \begin{itemize}
    \item Conducted extensive surveillance camera tests, validating the cameras' electrical, mechanical, and optical performance
    \item Engineered Python-based software for test automation and data analysis, enhancing test efficiency with up to 90\% automation
    \item Utilized FFmpeg for video processing and analysis
    \item Collaborated within a team using Git, GitHub, Confluence, and Jira to optimize workflow efficiency
    \end{itemize}
}

\section{Project}

\cventry
{Jan.-May 2024}
{Design, Simulation, and Analysis of a 10 Gbps Differential SerDes Data Link}
{}{}{}
{
    \begin{itemize}
        \item Designed a 10 Gbps differential SerDes data link using Cadence Virtuoso in 45 nm technology, incorporating a TX, a differential channel, and an RX. Performed circuit simulations using Spectre, achieving a power consumption of 9 mW, an eye-opening of 47 mV under the worst-case data pattern, and zero errors with PRBS7 input over 10,000 UIs.
        \item Characterized channel impulse and pulse responses using MATLAB, determining the worst-case data pattern and computing tap values for 2-tap FIR equalization.
        \item Developed the TX with a 4:1 serializer based on a 5-latch-2:1 multiplexer topology with half-rate clocking, requiring 5 GHz clock frequency at max. The pre-drivers are designed as inverter chains optimized using logical effort. Implemented a high-swing, voltage-mode driver segmented into 10 parts to provide a tunable impedance of $50 \Omega \pm 30\%$ and configured to pre-emphasize the signal, functioning as an FIR filter.
        \item Implemented the RX with 1:4 deserialization at 2.5 GHz using four slicers followed by synchronizers. Each slicer integrates a track-and-hold switch executed as a transmission gate, a StrongArm latch, and an SR latch.
    \end{itemize}
}


\cventry
{\textnormal{\textbf{Feb.-Mar. 2024}}}
{\textnormal{\textbf{AMBA AXI Stream Header Insertion}}}
{}{}{}
{
    \begin{itemize}
        \item Designed a Verilog RTL module for inserting headers into network packets, adhering to the AXI Stream protocol. The module receives headers and data packets from two master interfaces and sends them to one slave after packing and merging
        \item Implemented pipelining for both data and handshake signals, incorporating skid buffers to eliminate bubbles and efficiently manage backpressure
    \end{itemize}
}


\cventry
{\textnormal{\textbf{May-Oct. 2023}}}
{\textnormal{\textbf{Concurrent VLSI Routing with Multi-agent
Deep Reinforcement Learning}}}
{}{}{}
{
    \begin{itemize}
        \item Developed a Python-based machine learning framework to address the VLSI global routing problem in a concurrent manner. This framework modeled routing as a pathfinding task and solved it using multi-agent reinforcement learning integrated with deep neural networks (MLP and GNN implemented in PyTorch)
        \item Addressed training challenges by fine-tuning hyperparameters using grid search, leading to significant performance improvements
        \item The proposed work overcame the traditional net-ordering issue, guaranteed zero overflow, and outperformed an A* baseline by 2.6\% in terms of wirelength
        \end{itemize}
}

%\cventry
%{\textnormal{\textbf{Sept.-Dec. 2023}}}
%{\textnormal{\textbf{Computer Architecture %Simulator}}}
%{}{}{}
%{
%    \begin{itemize}
%        \item Optimized matrix multiplication in C by leveraging tiling and loop reordering techniques, altering memory access patterns and increasing cache hit rate by a factor of 8
%        \item Developed four cache replacement policies and two branch prediction algorithms in C++ in the ChampSim simulator
%    \end{itemize}
%}

\cventry
{\textnormal{\textbf{Jan.-Apr. 2023}}}
{\textnormal{\textbf{Microcomputer System Development}}}
{}{}{}
{
    \begin{itemize}
    \item Engineered key components of a microcomputer system including a 4-way set-associative cache and a DRAM controller using Verilog. The cache reduced the runtime of a benchmark by 43\% 
    \item Developed software in C that controlled a soft Motorola 68000 CPU to communicate with peripheral hardware (Flash, EEPROM, and ADC/DAC) using SPI, IIC, and CAN protocol 
    \item Utilized hardware timer interrupt and designed a snake game in C that ran on the system and interact with the player using VGA
    \end{itemize}
}

%\cventry
%{\textnormal{\textbf{Sept.-Dec. 2022}}}
%{\textnormal{\textbf{VLSI Physical Design}}}
%{}{}{}
%{
    %\begin{itemize}
    %\item Designed the physical layout of a 3-input NAND gate using Cadence Virtuoso, the layout was DRC error free and optimized for timing and area
    %\item Synthesized an RTL FSM using Cadence Encounter RTL Compiler
    %\item Placed and routed the post-synthesis netlist using Cadence Innovus
    %\item Simulated the circuit after placement and routing using Cadence Virtuoso 
%\end{itemize}
%}

\cventry
{\textnormal{\textbf{June-Sept. 2022}}}
{\textnormal{\textbf{CPU Design and Assembly programming}}}
{}{}{}
{
    \begin{itemize}
    \item Architected a 16-bit RISC CPU from the ground up using Verilog, integrating pivotal components such as FSM, datapath, RAM, and I/O interfaces. The CPU was implemented on an Altera FPGA
    \item The CPU supported 13 diverse instructions encompassing ALU operations, memory access, and branching mechanisms
    \item Implemented preemptive multitasking on an ARM core integrated on the FPGA using ARM Assembly
\end{itemize}
}

\cventry
{\textnormal{\textbf{Jan.-Apr. 2022}}}
{\textnormal{\textbf{FPGA SoC Design}}}
{}{}{}
{
    \begin{itemize}
    \item Developed a comprehensive SoC on FPGA using Quartus, incorporating a Nios II soft processor, on-chip RAM, and custom IP cores, managed via Avalon memory-mapped interface
    \item Designed and synthesized custom IP cores in Verilog for functionalities like image data processing, arithmetic acceleration and VGA output
    \item Created software in C to perform system tasks such as performance benchmarking and control over VGA display
    \end{itemize}
}

\section{Skills}
\cvitem{Hardware}{Verilog, SystemVerilog, FPGA, ModelSim, Quartus, Cadence}
\cvitem{Software}{Python $>$ C $>$ ARM Assembly = C++, Linux, MATLAB}
%\cvitem{DevOps Tools}{Confluence, Jira, Git, GitHub}
%\cvitem{SoC design}{Familiar with concepts of cache coherency, DVFS, DFT, ATPG, P\&R from courses}
% ... add more sections and subsections as needed ...
\end{document}
