\documentclass[11pt,a4paper,sans]{moderncv}

% moderncv themes
\moderncvstyle{banking}
\moderncvcolor{blue}

% adjust the page margins
\usepackage{lmodern}
\usepackage[scale=0.8, top=1cm, bottom=1cm, left=1cm, right=1cm]{geometry}

% personal data
\name{Bruce (Shidi)}{Xi}
\phone[mobile]{(236)-777-8218}
\email{brucexi99@outlook.com}
\social[linkedin]{bruce-shidi-xi}
\social[github]{brucexi999}

%----------------------------------------------------------------------------------
%            content
%----------------------------------------------------------------------------------
\begin{document}
\makecvtitle

\section{Education}
\cventry{2021--2024}{Master of Engineering in Electrical and Computer Engineering}{The University of British Columbia}{Vancouver, BC}{}{
    \begin{itemize}
    \item Relevant courses: Deep Learning, DNN Hardware Accelerator, Computer Architectures, Digital/Microcomputer System Design, VLSI, IC Testing and Reliability
\end{itemize}
}  % arguments 3 to 6 can be left empty
\cventry{2018--2021}{Bachelor of Engineering in Materials Science and Engineering}{Imperial College London}{London, UK}{}{
    \begin{itemize}
        \item Graduated with First-Class Honours
        \item Obtained Dean's List for three consecutive years (2018-2021)
    \end{itemize}
}

\section{Project}

\cventry
{\textnormal{\textbf{May-Oct. 2023}}}
{\textnormal{\textbf{Concurrent VLSI Routing with Multi-agent
Deep Reinforcement Learning}}}
{}{}{}
{
    \begin{itemize}
        \item Independently mastered reinforcement learning and VLSI global routing through  self-study and literature review
        \item Developed a novel machine learning framework to address the VLSI global routing problem in a concurrent manner, integrating multi-agent reinforcement learning with deep neural networks
        \item Addressed training challenges by fine-tuning hyperparameters through a grid search approach, improving performances
        \item Actively contributed to research group meetings by sharing project insights and progress, effectively communicating complex technical details to supervisors and peers
        \item The proposed work overcame the traditional net-ordering issue, guaranteed zero overflow, and outperformed an A* baseline by 2.6\% in terms of wirelength
        \end{itemize}
}

\cventry
{\textnormal{\textbf{Jan.-Apr. 2023}}}
{\textnormal{\textbf{Microcomputer System Development}}}
{}{}{}
{
    \begin{itemize}
    \item Engineered key components of a microcomputer system including a 4-way set-associative cache and a DRAM controller using Verilog. The cache reduced the runtime of a benchmark by 43\%
    \item Implemented the system on an Altera FPGA with a soft core provided by the course 
    \item Developed software in C that interacted with hardware (Flash, EEPROM, and ADC/DAC) using SPI, IIC, and CAN protocol 
    \item Utilized hardware timer interrupt and designed a snake game software that ran on the system and interact with the player using VGA
    \end{itemize}
}

\cventry
{\textnormal{\textbf{Jan.-Apr. 2022}}}
{\textnormal{\textbf{FPGA SoC Design}}}
{}{}{}
{
    \begin{itemize}
    \item Developed a comprehensive SoC on FPGA using Quartus, incorporating a Nios II soft processor, on-chip RAM, and custom IP cores, managed via an Avalon memory-mapped interface
    \item Designed and synthesized custom IP cores in Verilog for functionalities like image data processing, arithmetic acceleration and VGA output
    \item Created embedded software in C to perform system tasks such as performance benchmarking and control over VGA display
    \end{itemize}
}

%\cventry
%{\textnormal{\textbf{June-Sept. 2022}}}
%{\textnormal{\textbf{CPU Architecture Design}}}
%{}{}{}
%{
%    \begin{itemize}
%    \item Architected a 16-bit RISC CPU from the ground up using Verilog, integrating pivotal components such as FSM, datapath, RAM, and I/O interfaces
%    \item The CPU supported 13 diverse instructions encompassing ALU operations, memory access, and branching mechanisms
%    \item Successfully deployed the system onto an FPGA and validated the design's capabilities by executing a test program
%\end{itemize}
%}

\section{Experience}
\cventry{May-Dec. 2022}{Design Validation Co-op}{Motorola Solutions}{Vancouver, BC}{}
{
    \begin{itemize}
    \item Conducted extensive camera tests, ensuring precision both in lab settings and office environments
    \item Developed Python-based software, realizing test automation and data analysis, resulting in a significant enhancement in test efficiency. Some tests achieved automation of up to 90\%
    \item Collaborated effectively within a team framework, leveraging tools like Git and Jira for optimal workflow management 
    \end{itemize}
}
\cventry{June-Aug. 2021}{Undergraduate Research Assistant}{University College London}{London, UK}{}
{
    \begin{itemize}
    \item Played an integral role in the research team by meticulously taking measurements and preparing samples
    \item Demonstrated analytical skills by independently evaluating vast datasets and presenting insights effectively to the research group, fostering informed decision-making
    \end{itemize}
}

\section{Skills}
\cvitem{Hardware}{Verilog, FPGA, Modelsim, Quartus, Cadence}
\cvitem{Software}{Python $>$ C $>$ ARM Assembly = C++, Linux}
\cvitem{Research}{LaTeX, Academic writing, Mendeley}
\cvitem{}{}

% ... add more sections and subsections as needed ...
\end{document}
