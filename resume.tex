\documentclass[11pt,a4paper,sans]{moderncv}

% moderncv themes
\moderncvstyle{banking}
\moderncvcolor{blue}

% adjust the page margins
\usepackage{lmodern}
\usepackage[scale=0.8, top=0.5cm, bottom=1cm, left=1cm, right=1cm]{geometry}

% personal data
\name{Bruce}{Xi}
\phone[mobile]{(236)-777-8218}
\email{brucexi99@outlook.com}
\social[linkedin]{bruce-shidi-xi}
\social[github]{brucexi999}
%\extrainfo{\textbf{Available from May 2024 - August 2025}}

%----------------------------------------------------------------------------------
%            content
%----------------------------------------------------------------------------------
\begin{document}
\makecvtitle
\vspace{-1.5cm} 

\section{Education}
\cventry{2021--2024}{Master of Engineering in Electrical and Computer Engineering}{The University of British Columbia}{Vancouver, BC}{GPA 4.0}{
    \begin{itemize}
    \item Relevant courses: Digital/Microcomputer System Design, Computer Architecture, VLSI, High-speed Data Link, IC Testing and DFT, DNN Accelerator, Signals and Systems, Deep Learning
\end{itemize}
}
\cventry{2018--2021}{Bachelor of Engineering in Materials Science and Engineering}{Imperial College London}{London, UK}{First-Class Honours Degree}{
    \begin{itemize}
    %    \item Graduated with First-Class Honours
        \item Obtained Dean's List for three consecutive years (2018-2021)
    \end{itemize}
}



\vspace{-0.3cm} 

\section{Projects}

\cventry
{\textnormal{\textbf{May-Aug. 2024}}}
{\textnormal{\textbf{SLM Wear-out Monitoring IP Design Verification}}}
{}{}{}
{
    \begin{itemize}
        \item Created the test plan and feature list for a digital IP based on its specification. Adopted a bottom-up approach to test the RTL design using functional simulation, progressing from the unit level to the IP level.
        \item Developed modular and reusable testbench components (e.g., drivers, monitors, scoreboards), which were integrated into hierarchical verification environments implemented in SystemVerilog based UVM.
        \item Wrote UVM test sequences, including constrained random test cases and directed test cases, to thoroughly test the DUT in both normal and corner/error scenarios.
        \item Debugged failures in simulation waveforms, identified root causes, and collaborated with design engineers to resolve issues in the RTL code.
    \end{itemize}
}


\cventry
{\textnormal{\textbf{Feb.-Apr. 2024}}}
{\textnormal{\textbf{AMBA AXI Stream Header Insertion IP Design}}}
{}{}{}
{
    \begin{itemize}
        \item Designed an RTL IP using Verilog to insert headers into network data packets. This IP transfers data following the AXI Stream protocol, accepting packets and headers from two input interfaces. It removes invalid bytes from both input data based on the keep signals (packing), merges the remaining valid bytes, sends the processed data to the output ports, and outputs the corresponding keep and last signals.
        \item Used pipelining for data and handshake signals, and equipped skid buffers to handle backpressure, eliminate bubbles in the pipeline, and optimize throughput.
        %\item Verified the IP using constrained-random testing in SystemVerilog to ensure data correctness.
    \end{itemize}
}


\cventry
{\textnormal{\textbf{Jan.-May 2024}}}
{\textnormal{\textbf{Design of a 10 Gbps Differential SerDes Data Link with TX FIR Equalization}}}
{}{}{}
{
    \begin{itemize}
        \item Designed a 10 Gbps differential SerDes data link using Cadence Virtuoso in 45 nm technology, including a TX, differential channel, and RX. Achieved 9 mW power consumption, 47 mV eye-opening under worst-case data pattern, and zero errors with both worst-case and PRBS7 input over 10,000 UIs.
        \item Characterized channel responses using MATLAB, determined worst-case data pattern and the tap values for 2-tap FIR equalization.
        \item Developed a 4:1 serializer TX with a 5 GHz clock, featuring a high-swing, voltage-mode driver with tunable impedance and pre-emphasis FIR equalization for transmission line impedance matching and ISI suppression respectively.
        \item Implemented a 1:4 deserialization RX at 2.5 GHz using slicers with track-and-hold switches, StrongArm latches, and SR latches.
    \end{itemize}
}

\cventry
{\textnormal{\textbf{Jan.-Apr. 2023}}}
{\textnormal{\textbf{Microcomputer System Development}}}
{}{}{}
{
    \begin{itemize}
    \item Developed critical components for a microcomputer system capable of operating at 45 MHz on an Altera FPGA. Implemented an 8-way set associative cache using SRAM on the FPGA, and designed a segment of the cache controller featuring a Tree PLRU replacement policy using Verilog. Also designed components of a DRAM controller using Verilog.
    \item Wrote embedded C drivers to manage a soft Motorola 68000 microprocessor for communications via SPI protocol with a Flash, and via IIC protocol with a EEPROM and an ADC/DAC PCB.
    \item Leveraged hardware timer interrupts to create a snake game in C, which was playable on the system.
    \end{itemize}
}

%\cventry
%{\textnormal{\textbf{May-Oct. 2023}}}
%{\textnormal{\textbf{Concurrent VLSI Routing with Multi-agent Deep Reinforcement Learning}}}
%{}{}{}
%{
    %\begin{itemize}
        %\item Developed a Python-based machine learning model to address the VLSI global routing problem in a concurrent manner. Modeled routing as a pathfinding task and solved it using multi-agent reinforcement learning integrated with deep neural networks (MLP and GNN implemented in PyTorch).
        %\item Addressed training challenges by fine-tuning hyperparameters using grid search, leading to significant performance improvements
        %\item The proposed work overcame the traditional net-ordering issue, guaranteed zero overflow, and outperformed an A* baseline by 2.6\% in terms of wirelength.
        %\end{itemize}
%}

%\cventry
%{\textnormal{\textbf{Sept.-Dec. 2023}}}
%{\textnormal{\textbf{Computer Architecture %Simulator}}}
%{}{}{}
%{
%    \begin{itemize}
%        \item Optimized matrix multiplication in C by leveraging tiling and loop reordering techniques, altering memory access patterns and increasing cache hit rate by a factor of 8
%        \item Developed four cache replacement policies and two branch prediction algorithms in C++ in the ChampSim simulator
%    \end{itemize}
%}



%\cventry
%{\textnormal{\textbf{Sept.-Dec. 2022}}}
%{\textnormal{\textbf{VLSI Physical Design}}}
%{}{}{}
%{
    %\begin{itemize}
    %\item Designed the physical layout of a 3-input NAND gate using Cadence Virtuoso, the layout was DRC error free and optimized for timing and area. Parasitic extraction was done for the purpose of timing analysis and circuit simulation.
    %\item Synthesized an RTL state machine using Cadence Encounter RTL Compiler. Placed and routed the post-synthesis netlist using Cadence Innovus. Simulated the circuit after placement and routing using Spectre.
%\end{itemize}
%


%\cventry
%{\textnormal{\textbf{June-Sept. 2022}}}
%{\textnormal{\textbf{CPU Design and Assembly Programming}}}
%{}{}{}
%{
    %\begin{itemize}
    %\item Designed a 16-bit RISC CPU from the ground up using Verilog and implemented on an Altera FPGA, integrating pivotal components such as a state machine, a datapath, on-chip memory, and I/O interfaces.
    %\item The CPU supported 13 diverse instructions encompassing ALU operations, memory access, and branching mechanisms.
    %\item Implemented preemptive multitasking on an ARM Cortex-A9 core embedded in the FPGA using ARM Assembly.
%\end{itemize}
%}

%\cventry
%{\textnormal{\textbf{Jan.-Apr. 2022}}}
%{\textnormal{\textbf{FPGA SoC Design}}}
%{}{}{}
%{
    %\begin{itemize}
    %\item Developed a System-on-Chip running at 50 MHz on an Altera FPGA using Quartus IP and Verilog, integrating a Nios II soft processor, on-chip RAM, VGA core, JTAG UART core for host PC communication, and custom IPs. Utilized the Avalon Memory-Mapped Interface for module communication.
    %\item Designed and implemented custom IPs, including counters, state machines, and Avalon interfaces. Conducted simulations of individual IPs and the complete system using ModelSim to verify functionality.
    %\item Created embedded software in C to interact with system modules via memory-mapped addresses, enabling data read/write operations with IPs and performing arithmetic operations on the CPU.
    %\end{itemize}
%}

\vspace{-0.2cm} 

\section{Work Experience}
\cventry{May-Dec. 2022}{Design Validation Co-op}{Motorola Solutions}{Vancouver, BC}{}
{
    \begin{itemize}
    \item Conducted electrical and optical testing on surveillance cameras by following a standard workflow, documented results, and managed issues using Confluence and Jira.
    \item Analyzed test results to identify performance trends and quality issues, showcasing analytical skills and attention to detail, leading to early detection of potential defects and reducing post-production errors.
    \item Proposed improvements and innovations in existing testing procedures, self-taught Python programming to develop software for test automation and data analysis, utilizing OOP to create modular, reusable scripts. Successfully reduced repetitive manual tasks, increasing team efficiency by automating tests up to 90\%, demonstrating problem-solving and debugging skills.
    \item Collaborated with electrical and firmware teams to identify and resolve hardware and software issues, showcasing teamwork and communication skills.
    \end{itemize}
}

\vspace{-0.2cm} 

\section{Skills}
\cvitem{Hardware}{Verilog, SystemVerilog, UVM, Altera FPGA, ModelSim, Quartus, Vivado, Cadence Virtuoso} 
\cvitem{Software}{Python, OOP, C, ARM Assembly, Linux, MATLAB}
%\cvitem{DevOps Tools}{Confluence, Jira, Git, GitHub}
%\cvitem{SoC design}{Familiar with concepts of cache coherency, DVFS, DFT, ATPG, P\&R from courses}
% ... add more sections and subsections as needed ...
\end{document}
