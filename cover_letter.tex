\documentclass[a4paper,12pt]{letter}
\usepackage[utf8]{inputenc}
\usepackage{geometry}

% Adjust the page margins
\geometry{top=1.0in, bottom=1.0in, left=1.0in, right=1.0in}

\begin{document}

\begin{letter}{MASc Oppportunities}

\opening{Dear Prof. Moshovos,}

I'm writing to provide further insight into my background and aspirations, which I could not fully elaborate on within the constraints of the email. I'd like to begin by explaining the reasoning behind my interest in deep learning accelerators.

Machine learning (ML), particularly deep learning, is undeniably a big deal these days as many are talking about it and are drawn to this domain by its transformative potential and its promising career prospects. I am also one of these people. However, my primary passion is rooted in hardware, which is the driving force behind my transition from a materials science background, with a focus on semiconductors, to ECE. My journey in this field began with foundational undergraduate courses and I have since advanced to more specialized topics. And I am determined to develop my future career as a hardware engineer. This is why I am not joining the software side of the ML spectrum. Instead, working on hardware accelerators for deep learning seems like a perfect combination of my dream and the practicalities of the evolving tech landscape. The interdisciplinary nature of this subject has captivated me since my initial exposure to it.

Beyond my enthusiasm for the subject lies a set of competencies that I believe position me well for success in research. As a student, I am always committed to excellence and take things seriously. Regardless of the task at hand, I immerse myself in thorough research, meticulous preparation, and diligent execution. Before my weekly research group meetings, I invest time in drafting notes and revisiting them multiple times to ensure clarity and thoroughness in my reports. While seemingly minor, I believe it is such attention to detail that shapes one's overall trajectory. Furthermore, my collaborative spirit extends beyond my individual responsibilities. Last summer, in addition to my own research, I assisted a Ph.D. student in the group with coding for an integrated circuit test simulator. I believe a supportive atmosphere built by supportive people is essential in every research group. 

In conclusion, my keen interest, coupled with my capabilities, makes me confident in my potential contribution to your esteemed research group. If there are indeed openings in the team, I would really appreciate your feedback, and I will list your name in the application form. On a final note, please accept my congratulations on the recognition of your Cnvlutin paper as part of the International Symposium on Computer Architecture's 50th Retrospective collection. I had the chance to review this paper as classwork and it has convinced me of the power of exploiting sparsity for DNN accelerators.

\closing{Best wishes,

Bruce Xi}

\end{letter}
\end{document}
